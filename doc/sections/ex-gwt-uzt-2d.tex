\section{Two-Dimensional Test of Unsaturated-Saturated Transport}

This example first appeared as scenario 6 in \cite{morway2013}.  At that time, capabilites were added to MT3DMS \citep{zheng1999mt3dms} for simulating solute transport in the unsaturated-zone using flux terms calculated by the unsaturated-zone flow (UZF1; \cite{UZF}) package. \cite{morway2013} referred to the published MT3DMS variant as UZF-MT3DMS. Eventually, however, these capabilities were better documented and released with MT3D-USGS \citep{mt3dusgs}. For the purpose of testing unsaturated-zone transport using the UZF/UZT packages inside \mf, the \mf solution is compared against the MT3D-USGS solution. Moreover, we note that the results of the  MT3D-USGS simulation were compared to results calculated by VS2DT \citep{lappalaetal1987VS2D} for establishing the accuracy of the MT3D solution.  VS2DT solves Richards' equation and as such can simulate flow and solute fluxes across the unsaturated-saturated interface.  Therefore, this example problem tests the ability of \mf to accurately simulate the infiltration, unsaturated-zone transport, recharge, and subsequent saturated transport of dissolved solute.

\subsection{Example description}

For this problem, a relatively small, two-dimensional profile that is 10 $m$ wide by 5 $m$ deep is used.  Layer thickness, as well as the column widths, are 0.25 $m$. A constant head boundary of 1.625 $m$ is set on the left and right sides of the active model domain.  An infiltration rate of 0.1 $m/day$ is specified all along the top boundary except for the left- and right-most columns where the constant head boundary condition exists.  A no-flow boundary is used along the bottom of the simulation domain.   Additional model parameter values are listed in table~\ref{tab:ex-gwt-uzt-2d-01}.  

% add 2nd static parameter value table
\input{../tables/ex-gwt-uzt-2d-01}

Given the relatively dry initial condition within the unsaturated zone, the infiltrating front reaches the water table on day 8 of the 60 day simulation period. Once the infiltrating wave reaches the saturated zone, the water table rises into the unsaturated zone and further tests the accuracy of the transport solution.

\subsection{Example results}

Because \mf does not (yet) simulate dispersion in the unsaturated-zone, there are some significant differences between the two solutions (figure~\ref{fig:ex-gwt-uzt-2d}).  Whereas longitudinal and transverse dispersive fluxes spread solute ahead and to the side of the downward migrating plume within MT3D-USGS, the UZF/UZT formulation within \mf simulate pure advective transport.  However, once solute reaches the saturated zone, dispersion is simulated using the XT3D package (the default setting).  

% Morway et al. (2013) figure 6
\begin{StandardFigure}
	{A two-dimensional problem first published in \cite{morway2013}.  MT3D-USGS results closely match a benchmark solution calculated by VS2DT \citep{lappalaetal1987VS2D}.}
	{fig:ex-gwt-uzt-2d}
	{../figures/ex-gwt-uzt-2d.png}
\end{StandardFigure}
