\section{Drain Package Drainage Depth Option Problem 1}

% Describe source of problem
This example is a modified version of the Unsaturated Zone Flow (UZF) Package problem 2 described in \cite{UZF}. UZF Package problem 2 is based on the Green Valley problem (Streamflow Routing (SFR) Package problem 1) described in \cite{modflowsfr1pack}. The problem has been modified by converting all of the SFR reaches to use rectangular channels and to use the drain package drainage option to simulated groundwater discharge to the land surface.                               

\subsection{Example Description}
% spatial discretization  and temporal discretization
Model parameters for the example are summarized in table~\ref{tab:ex-gwf-drn-p01-01}.  The model consists of a grid of 10 columns, 15 rows, and 1 layer. The model domain is  50,000 $ft$ and 80,000 $ft$ in the x- and y-directions, respectively. The discretization is 5,000 $ft$ in the row and column direction for all cells. The top of the model ranges from about 1,000 to 1,100 $ft$ and the bottom of the model ranges from about 500 to 1,000 $ft$.

Twelve stress periods are simulated. The first stress period is steady state and the remaining stress periods are transient. The stress periods are $2.628 \times 10^{6}$ seconds (30.42 days) in length. The first stress period is broken into one time step. Stress periods 2 through 12 are each broken up into 15 time steps and use a time step multiplier of 1.1.

% add static parameter table(s)
\input{../tables/ex-gwf-drn-p01-01}

% material properties
The basin fill thickens toward the center of the valley and hydraulic conductivity of the basin fill is highest in the region of the stream channels. Hydraulic conductivity is 173 $ft/day$ ($2 \times 10^{-4}$ $ft/s$) in the vicinity of the stream channels and 35 $ft/day$ ($4 \times 10^{-4}$ $ft/s$) elsewhere in the alluvial basin. A constant specific storage value of $1 \times 10^{-6}$ ($1/day$) was specified throughout the alluvial basin. Specific yield is 0.2 (unitless) in the vicinity of the stream channels and 0.1 (unitless) elsewhere in the alluvial basin.

% initial conditions
An initial head of 1,050 $ft$ was specified in all model layers. Any initial head exceeding the bottom of each cell could be specified since the model is steady-state.

% boundary conditions
Flow into the system is from infiltration from precipitation and was represented using the UZF package. Recharge rates applied to each cell ranged $2.5 \times 10^{-10}$ to $2 \times 10^{-9}$ from  of $3 \times 10^{-7}$ $ft/s$, with lower rates in the vicinity of the stream channels and higher rates elsewhere in the alluvial basin. Flow out of the model is from groundwater evapotranspiration represented by evapotranspiration (EVT) package cells and discharging wells represented by well (WEL) package cells. Groundwater evapotranspiration occurs where depth to water is within 15 $ft$ of land surface, has a maximum rate of 3 $ft/yr$ at land surface, and is coincident with the valley lowland through which several streams flow. Wells are  located in ten cells (rows 6 through 10 and columns 4 and 5) along the west side of the valley (fig.~\ref{fig:ex-gwf-drn-p01-grid}\textit{B}). Each well extracted 10 $ft^{3}/s$ of groundwater for a total withdrawal rate of 100 $ft^{3}/s$ (about twice the steady-state ground-water inflow). Two general-head boundary cells were added in (row 13, column 1) and (row 14, column 8) with a specified head equal to 988 and 1,045 $ft$, respectively, and a constant conductance of 0.038 $ft^{2}/s$.

\input{../tables/ex-gwf-drn-p01-02}

The streams in the model domain were represented using a total of 36 reaches. External inflows of 25, 10, and 100 $ft^{3}/s$ were specified for reach 1, 16, and 28, respectively. Reach 1 is located in (row 1, column 1), reach 16 is in (row 5, column 10), and reach 28 is in (row 14, column 9). Streamflow discharges from the model at the downstream end of reach 36 in (row 13, column 1). Reach widths were specified to be 12, 0, 5, 12, 55, and 40 $ft$ for reaches 1--9, 10--18, 19--22, 23--27, 28--30, and 31--36, respectively. The remaining streambed properties and stream dimensions used for each stream reach are the same as those used in 
 \cite{modflowsfr1pack} \cite[see][Table~1]{modflowsfr1pack}. Constant stage reaches were used to define the ditch represented by reaches 10--15 and ranged from approximately 1,075.5--1061.6 $ft$. A diversion from reach 4 to 10 was specified to represent managed inflows to the ditch. Ditch inflows were specified to be 10 $ft^{3}/s$ except if the downstream flow in reach 4 is less than the specified diversion rate; in cases where the downstream flow in reach 4 is less than the specified diversion rate all of the downstream flow in reach 4 is diverted to the ditch and the inflow to reach

% solution 
The model uses the Newton-Raphson Formulation. The simple complexity Iterative Model Solver option and preconditioned bi-conjugate gradient stabilized linear accelerator is also used.

% for examples without scenarios
\subsection{Example Results}

Simulated results for the initial steady-state stress period and at the end of the stress period with groundwater pumping (stress period 2) are shown in figure~\ref{fig:ex-gwf-drn-p01-01}. Reach stage and downstream discharge were also evaluated for reach 4, 14, 27, and 36.

% a figure
\begin{StandardFigure}{
                                     Simulated water levels and normalized specific discharge vectors  
                                     under steady state and pumping conditions. 
                                     \textit{A}. steady-state results.
                                     \textit{B}. results after 50 years of pumping.
                                     }{fig:ex-gwf-drn-p01-01}{../figures/ex-gwf-drn-p01-01.png}
\end{StandardFigure}                                 


