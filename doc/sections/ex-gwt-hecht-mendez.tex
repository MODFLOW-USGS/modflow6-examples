\section{Hecht-Mendez 3D Borehole Heat Exchanger Problem}

% Describe source of problem
The models presented in \cite{hechtMendez2010} apply MT3DMS \citep{zheng1999mt3dms} as a heat transport simulator.  Both 2-dimensional and 3-dimensional demonstration problems are presented that explore the use of a ``borehole heat exchanger'' (BHE), a ``closed'' geothermal system that uses a heat pump for cycling water and anti-freeze fluids in pipes for mining heat from an aquifer \citep{diao2004}.  Figure~\ref{fig:hechtMendezBHE} depicts the kind of system modeled in this example.

Among the examples presented in \cite{hechtMendez2010}, this work recreates the 3D example that simulates heat exchange between a BHE and the aquifer it is placed in.  Among the suite of examples included in the MODFLOW6-examples.git repo, this is the first one demonstrating the suitability of the GWT model within \mf for simulating saturated zone heat transport.  To verify the applicability of \mf6 as a groundwater heat transport simulator, we compare the \mf solution to established analytical solutions.  The analytical solutions are described in more detail below.

% a figure
\begin{StandardFigure}{
                                     Example of a closed ground source heat pump extracting heat beneath a private residence.  Image taken from \cite{hecht2008}.
                                     }{fig:hechtMendezBHE}{../images/hechtMendezBHE.png}
\end{StandardFigure}

\subsection{Example description}

Through appropriate substitution of heat-related transport terms into the groundwater solute transport equation, \mf (as is the case with MT3DMS and MT3D-USGS) may be used as a heat simulator for the saturated zone by observing that the heat transport equation,

\begin{equation*}
\theta \rho_w C_w \frac{\partial T}{\partial t} + \left( 1 - \theta \right) \rho_s C_s \frac{\partial  T_s}{\partial t} = div \left( \left( k_{T_m} + \theta \rho_w C_w \alpha_h \upsilon_a \right) grad T\right) - div \left( \theta \rho_w C_w \upsilon_a T \right) + q_h
\end{equation*}

has a similar form as the groundwater solute transport equation,

\begin{equation*}
\left[ 1 + \frac{\rho_b K_d}{\theta} \right] \theta \frac{\partial  C}{\partial t}
= div \left[ \theta \left( D_m + \alpha \upsilon_a \right) grad C \right] - div \left( \upsilon_a \theta C \right) + q_s C_s - \lambda \theta C
\end{equation*}

Originally, \cite{hechtMendez2010} included three sub-scenarios for the 3D test model.  The first sub-scenario used a Peclet number of 0, indicating no velocity and therefore explored a purely conductive environment.  Owing to the fact that \citep{hechtMendez2010} did not publish results for this particular sub-scenario and because a 3D analytical solution for a purely conductive problem is not available, neither is this sub-scenario explored here.  The two remaining sub-scenarios investigate the effect of groundwater velocity on the heat profile down-gradient of the BHE.  To accomplish this goal, Peclet numbers of 1.0 and 10.0 are explored. A Peclet number of 1.0 indicates an approximate balance between convective (i.e., advective) and conductive heat transport.  Meanwhile, a Peclet number of 10.0 represents a convection-dominated transport environment.  The parameters used in each of the two scenarios presented are listed in table~\ref{tab:ex-gwt-hecht-mendez-scenario}.

% add scenario summary parameter table(s)
\input{../tables/ex-gwt-hecht-mendez-scenario}
