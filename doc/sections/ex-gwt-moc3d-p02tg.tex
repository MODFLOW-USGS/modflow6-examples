\section{MOC3D Problem 2 with Triangular Grid}

% Describe source of problem
This problem corresponds to the second problem presented in the MOC3D report \cite{konikow1996three}, which involves the transport of a dissolved constituent in a steady, three-dimensional flow field.  An analytical solution for this problem is given by \cite{wexler1992}.  As for the previous example, this example is simulated with the GWT Model in \mf, which receives flow information from a separate simulation with the GWF Model in \mf.  In this example, however, a triangular grid is used for the flow and transport simulation.  Results from the GWT Model are compared with the results from the \cite{wexler1992} analytical solution.

% a figure
\begin{StandardFigure}{
                                     Triangular model grid used for the \mf simulation.  Model grid is shown using an aspect ratio of 4.
                                     }{fig:ex-gwt-moc3d-p02tg-grid}{../figures/ex-gwt-moc3d-p02tg-grid.png}
\end{StandardFigure}            


\subsection{Example description}

\cite{wexler1992} presents an analytical solution for three dimensional solute transport from a point source in a one-dimensional flow field.  As described by \cite{konikow1996three}, only one quadrant of the three-dimensional domain is represented by the numerical model.  Thus, the solute mass flux specified for the model is one quarter of the solute mass flux used in the analytical solution.  

The parameters used for this problem are listed in table~\ref{tab:ex-gwt-moc3d-p02tg-01}.  The model grid for this problem consists of 40 layers, 695 cells per layer, and 403 vertices in a layer.  The top for layer 1 is set to zero, and flat bottoms are assigned to all layers based on a uniform layer thickness of 0.05 $m$.  The remaining parameters are set similarly to the previous simulation with a regular grid, except for in this simulation, the XT3D method is used for flow and dispersive transport.

% add static parameter table(s)
\input{../tables/ex-gwt-moc3d-p02tg-01}

% for examples without scenarios
\subsection{Example Results}

A comparison of the MODFLOW 6 results with the analytical solution of \cite{wexler1992} is shown for layer 1 in figure~\ref{fig:ex-gwt-moc3d-p02tg-map}.

% a figure
\begin{StandardFigure}{
                                     Concentrations simulated by the \mf GWT Model and calculated by the analytical solution for three-dimensional flow with transport.  Results are for the end of the simulation (time=400 $d$) and for layer 1.  Black lines represent solute concentration contours from the analytical solution \citep{wexler1992}; blue lines represent solute concentration contours simulated by \mf.  An aspect ratio of 4.0 is specified to enhance the comparison.
                                     }{fig:ex-gwt-moc3d-p02tg-map}{../figures/ex-gwt-moc3d-p02tg-map.png}
\end{StandardFigure}            

                
