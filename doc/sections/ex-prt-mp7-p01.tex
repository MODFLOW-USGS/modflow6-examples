\section{PRT/MP7 Example 1: Particle Tracking on a Structured Grid with Steady-State Flow}

\subsection{Example description}

This example demonstrates particle-tracking with a steady-state flow system consisting of two aquifers separated by a thin low-conductivity confining layer. The example is reproduced from the MODPATH 7 example problems. A schematic is shown in figure~(fig~\ref{fig:ex-prt-mp7-p01-config}). MODFLOW 6 PRT and MODPATH 7 simulations are run, then solutions are compared side-by-side.

\begin{StandardFigure}{
    Conceptual model, reproduced from the MODPATH 7 examples document.
    }{fig:ex-prt-mp7-p01-config}{../images/ex-prt-mp7-p01-config.png}
\end{StandardFigure}

The system is simulated with a traditional structured model grid consisting of 3 model layers, 21 rows, and 20 columns. Areal grid cells are uniform square cells, 500 feet per side. The well is located in row 11, column 10, and layer 3. The river is located in layer 1, column 20, rows 1 - 21.

\subsection{Example Results}

The head simulated by the MODFLOW 6 groundwater flow model is shown in figure~(fig~\ref{fig:ex-prt-mp7-p01-grid-head}).

\begin{StandardFigure}{
    Simulated head.
    }{fig:ex-prt-mp7-p01-grid-head}{../figures/mp7-p01-grid-head.png}
\end{StandardFigure}

Particle tracking is often used to compute recharge capture areas for wells and other hydrologic features. In simulation 1A, a line of 21 particles is placed at the water table in layer 1 for column 3, rows 1 through 21. In simulation 1B, a denser release configuration is used which places a 3 x 3 array of particles on the top face of every cell in layer 1. In both simulations, particles are tracked forward to their discharge points. Pathlines are colored by discharge area (well or river) in figure~(fig~\ref{fig:ex-prt-mp7-p01-paths}).

\begin{StandardFigure}{
    Particle pathlines, colored by destination: particles with red pathlines are captured by the well, particles with blue pathlines are captured by the river.
    }{fig:ex-prt-mp7-p01-paths}{../figures/mp7-p01-paths.png}
\end{StandardFigure}

In figure~(fig~\ref{fig:ex-prt-mp7-p01-rel-destination}), capture areas for the well and river are displayed by plotting the starting locations of all particles, color-coded according to the zone value of the cells in which they terminate.

\begin{StandardFigure}{
    Particle release points, colored by destination.
    }{fig:ex-prt-mp7-p01-rel-destination}{../figures/mp7-p01-rel-destination.png}
\end{StandardFigure}

Travel time analysis is also a common use case for particle tracking. In figure~(fig~\ref{fig:ex-prt-mp7-p01-rel-travel-time}), starting locations are colored by travel time.

\begin{StandardFigure}{
    Particle release points, colored by travel time.
    }{fig:ex-prt-mp7-p01-rel-travel-time}{../figures/mp7-p01-rel-travel-time.png}
\end{StandardFigure}