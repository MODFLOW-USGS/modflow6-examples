\section{MODPATH 7 Example 1: Structured Grid, Steady-State Flow}

This example's flow system consists of two aquifers separated by a thin low conductivity confining layer.

\begin{StandardFigure}{
    Conceptual model.
    }{fig:ex-prt-mp7-p01-config}{../images/ex-prt-mp7-p01-config.png}
\end{StandardFigure}

The system is simulated with a traditional structured model grid consisting of 3 model layers, 21 rows, and 20 columns. Areal grid cells are uniform square cells, 500 feet per side. The well is located in row 11, column 10, and layer 3. The river is located in layer 1, column 20, rows 1 - 21. The grid and boundary conditions are shown in figure~(fig~\ref{fig:ex-prt-p01-grid}).

\begin{StandardFigure}{
    Model grid and boundary conditions.
    }{fig:ex-prt-mp7-p01-grid}{../figures/mp7-p01-grid.png}
\end{StandardFigure}

In simulation 1A, a line of 21 particles is placed at the water table in layer 1 for column 3, rows 1 through 21. A pathline simulation is run that tracks the particles forward to their discharge locations.

\begin{StandardFigure}{
    Particle pathlines (1A)
    }{fig:ex-prt-mp7-p01-paths}{../figures/mp7-p01-paths.png}
\end{StandardFigure}

Particle tracking is often used to compute recharge capture areas for wells and other hydrologic features. Forward-tracking endpoint simulations are an efficient way to compute capture areas. In simulation 1B, a capture area for the well is determined by placing a 3 x 3 array of particles on the top face of layer 1 and then tracking the particles forward to their discharge points. The capture area for the well can be displayed by plotting the starting locations of all the particles color-coded according to the zone value of the final cells in which they terminate.

\begin{StandardFigure}{
    Particle 
    }{fig:ex-prt-mp7-p01-rel-capt}{../figures/mp7-p01-rel-capt.png}
\end{StandardFigure}

Simulation 1B has a denser release configuration than that used for simulation 1A. The starting locations of particles that discharge to the cell containing the well are shown in red. Starting locations that discharge to the river are shown in blue.

Travel time analysis is also a common use case for particle tracking. Starting locations can also be colored by travel time

\begin{StandardFigure}{
    Model grid and boundary conditions.
    }{fig:ex-prt-mp7-p01-term-tt}{../figures/mp7-p01-term-tt.png}
\end{StandardFigure}