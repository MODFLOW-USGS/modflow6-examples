\section{PRT/MP7 Example 1: Particle Tracking on a Structured Grid with Steady-State Flow}

This example demonstrates a MODFLOW 6 particle tracking (PRT) model by reproducing example problem 1 from the MODPATH 7 \citep{pollock2016modpath7} example problems document \citep{modpath7examples}. An equivalent MODPATH 7 model is constructed as well.

\subsection{Example description}

The example first runs a groundwater flow (GWF) model simulating steady-state flow on a structured grid. The flow system includes an upper and lower aquifer separated by a confining layer with lower conductivity. The grid has 3 layers, 21 rows, and 20 columns, with square cells 500 feet to a side. The system includes two boundary conditions: a well in layer 3, row 11, column 10, and a river in layer 1, column 20~(fig~\ref{fig:ex-prt-mp7-p01-config}). Model parameters for this example are summarized in table~\ref{tab:ex-prt-mp7-p01-01}.

\begin{StandardFigure}{
    Conceptual model. Image reproduced from the MODPATH 7 examples document \citep{modpath7examples}.
    }{fig:ex-prt-mp7-p01-config}{../images/ex-prt-mp7-p01-config.png}
\end{StandardFigure}

\input{../tables/ex-prt-mp7-p01-01.tex}

\subsection{Example Results}

In this example a MODFLOW 6 particle tracking (PRT) model runs in the same simulation as a groundwater flow (GWF) model~(fig~\ref{fig:ex-prt-mp7-p01-head}), which provides it with intercell flows via a GWF-PRT model exchange.

\begin{StandardFigure}{
    Heads simulated by the MODFLOW 6 groundwater flow (GWF) model.
    }{fig:ex-prt-mp7-p01-head}{../figures/mp7-p01-head.png}
\end{StandardFigure}

In subproblem 1A, a line of 21 particles is placed at the water table in layer 1 for column 3, rows 1 through 21. In subproblem 1B, a denser release configuration is used which places a 3 x 3 array of particles on the top face of every cell in layer 1. Both simulations track particles forward to their discharge points.

Subproblem 1A path points on a 1000-day time interval are visualized in 2D~(fig~\ref{fig:ex-prt-mp7-p01-paths-layer}) and 3D~(fig~\ref{fig:ex-prt-mp7-p01-paths-3d}).

\begin{StandardFigure}{
    Particle pathlines and 1000-day points for subproblem 1A. Points are colored by layer.
    }{fig:ex-prt-mp7-p01-paths-layer}{../figures/mp7-p01-paths-layer.png}
\end{StandardFigure}

\begin{StandardFigure}{
    Three-dimensional perspective of pathlines and 1000-day points for subproblem 1A. Points are colored by layer.
    }{fig:ex-prt-mp7-p01-paths-3d}{../figures/mp7-p01-paths-3d.pdf}
\end{StandardFigure}

To illustrate discharge points, pathlines are colored by discharge area (well or river) in fig~\ref{fig:ex-prt-mp7-p01-paths}. To show capture areas, starting locations of all particles are color-coded according to the zone value of the cells in which they terminate in fig~\ref{fig:ex-prt-mp7-p01-rel-destination}. Travel time analysis is also a common use case for particle tracking. Particle release points are colored by total travel time to capture in fig~\ref{fig:ex-prt-mp7-p01-rel-travel-time}.

\begin{StandardFigure}{
    Particle pathlines, colored by destination: particles with red pathlines are captured by the well, particles with blue pathlines are captured by the river.
    }{fig:ex-prt-mp7-p01-paths}{../figures/mp7-p01-paths.png}
\end{StandardFigure}

\begin{StandardFigure}{
    Particle release points, colored by destination.
    }{fig:ex-prt-mp7-p01-rel-destination}{../figures/mp7-p01-rel-destination.png}
\end{StandardFigure}

\begin{StandardFigure}{
    Particle release points, colored by travel time.
    }{fig:ex-prt-mp7-p01-rel-travel-time}{../figures/mp7-p01-rel-travel-time.png}
\end{StandardFigure}